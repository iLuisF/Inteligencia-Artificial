\documentclass{article}

\usepackage{amsfonts} 
\usepackage{amsmath} 
\usepackage{graphicx} 
\usepackage{float} 
\usepackage{natbib} 
\usepackage{slashbox} 
\usepackage{graphicx} 
\usepackage{flushend} 
\usepackage{amsmath} 
\usepackage{amssymb} 
\usepackage{amsxtra} 
\usepackage{amstext} 
\usepackage{amsthm} 
\usepackage{amsbsy} 
\usepackage{latexsym} 
\usepackage{mathrsfs} 
\usepackage{eucal} 
\usepackage{synttree} 


\usepackage[spanish]{babel}
\usepackage[utf8x]{inputenc}
\title{Espacio de estados}
\author{Flores González Luis Brandon}
\date{17 de Diciembre de 2017}


\begin{document}

	\maketitle
	Problema de los caníbales y los misionarios. Hay tres misionarios y tres caníbales a un lado del río y requieren cruzar al otro lado. Para ello cuentan con una pequeña barca donde sólo caben dos personas, aunque basta una para poder remar y cruzar el río Si en algún momento hay más caníbales que misionarios, los caníbales se comen a esos misionarios.
	
	\begin{enumerate}
		\item Define a las variables de este sistema y sus valores posibles $S$
		\\$Simbolos = {0, 1, 2, 3, M, C, B}$
		\\$S = {(M, C, B) | 0 \leq M \leq 3, 0 \leq C \leq 3, 0 \leq B \leq 1}$
		\\M es la cantidad de misionarios del lado derecho.
		\\C es la cantidad de canibales del lado derecho.
		\\B es la dirección de la barca, 0 esta del lado izquierdo y 1 del lado derecho.
		\item Define las acciones posibles $A$
		\\$A = mover(M, C, B)$ tal que $0 \leq M \leq 3, 0 \leq C \leq 3, 0 \leq B \leq 1$
		\item Define la función de transición 
		\\$\gamma = \gamma(s, mover(M, C, B)) -> s'$
		\\$s'$ es la suma o resta de los vectores $(1, 0, 1) (2, 0, 1) (0, 1, 1) (0, 2, 1) (1, 1, 1)$ si $ M \geq C$ cuando $M es diferente a 0$ si $M = 0$ entonces $M \leq C$ es valido.
		\\Además el vector (2, 1, 1) no es valido. 
		\\Por otra parte cualquier resultado fuera de los limites del problema (3, 3, 1) y (0, 0, 0). Serán inválidos
		\item ¿Cuántos estados tiene este espacio, de acuerdo a tu definición?
		\item Dibuja la cerradura transitiva de a partir del estado con los misionarios, los caníbales y la barca en el mismo lado del río.
		
		\synttree [(3, 3, 1)[(3, 2, 0) [(3, 3, 1)]]
		[(3, 1, 0)[(3, 2, 1) [(2, 2, 0)][(3, 1, 0)][(3, 0, 0)][(2, 1, 0)[(3, 1, 1)][(2, 2, 1)[(2, 1, 0)][(2, 0, 0)][(1, 1, 0)]][(3, 2, 1)]]][(3, 3, 1)]]
		[(2, 2, 0) [(3, 2, 1)][(3, 3, 1)]]] 
		
		\item ¿Qué solución encuentras para que todos crucen el río con vida?		
		\\$(3, 3, 1) (3, 1, 0) (3, 2, 1) (3, 0, 0) (3, 1, 1) (1, 1, 0) (2, 2, 1) (0, 2, 0) (0, 3, 1) (0, 1, 0) (0, 2, 1) (0, 0, 0)$
		\\También puede haber esta otra solución ligeramente diferente.
				\\$(3, 3, 1) (3, 1, 0) (3, 2, 1) (3, 0, 0) (3, 1, 1) (1, 1, 0) (2, 2, 1) (0, 2, 0) (0, 3, 1) (0, 1, 0) (1, 1, 1) (0, 0, 0)$
	\end{enumerate}
\end{document}